\section{Sprachmodelle und ihre Bedeutung}
\subsection{Abgrenzung Sprachmodelle}
\acp{llm} sind hochentwickelte KI-Systeme, die darauf spezialisiert sind, menschenähnlichen Text zu verarbeiten, zu verstehen und zu generieren. Sie basieren auf fortschrittlichen Deep-Learning-Techniken und werden durch umfangreiche Datensätze trainiert, die Milliarden von Wörtern aus verschiedenen Quellen wie Websites, Büchern und Artikeln enthalten. Durch dieses umfangreiche Training können LLMs die Nuancen von Sprache, Grammatik, Kontext und sogar einige Aspekte des Allgemeinwissens erfassen. % Hier könnte ein Zitat zu den Grundlagen von LLMs eingefügt werden.

Einige \acp{llm}, wie beispielsweise GPT-4 von OpenAI, nutzen eine spezielle Form des neuronalen Netzwerks, bekannt als Transformer. Diese Architektur ermöglicht es ihnen, komplexe sprachliche Aufgaben mit beeindruckender Kompetenz zu bewältigen. Ihre Anwendungen sind vielfältig und umfassen die Beantwortung von Fragen, die Erstellung von zusammenfassendem Text, die Übersetzung von Sprachen, die Generierung von Inhalten und die Führung von interaktiven Gesprächen mit Benutzern. % Hier könnte ein Zitat zu den Anwendungen von LLMs eingefügt werden.

Trotz der kontinuierlichen Weiterentwicklung von LLMs und ihres erheblichen Potenzials zur Verbesserung und Automatisierung verschiedener Anwendungen in Branchen wie Kundenservice, Inhaltserstellung, Bildung und Forschung, müssen auch ethische und gesellschaftliche Bedenken berücksichtigt werden. Dazu gehören mögliche voreingenommene Verhaltensweisen und Missbrauch, die im Rahmen des technologischen Fortschritts aktiv angegangen werden müssen. % Hier könnte ein Zitat zu den ethischen Bedenken von LLMs eingefügt werden.

\subsection{Einsatzgebiete}
LM-Technologien finden in verschiedenen Anwendungsfeldern vielfältige Verwendungsmöglichkeiten. Yang et al. haben sechs unterschiedliche Anwendungsszenarien identifiziert \footcite[Vgl.][S. 6 ff.]{yang2023harnessing}, darunter insbesondere solche im Bereich des Natural Language Understanding (NLU). Hierzu zählen beispielsweise Textklassifikationen, die einen Schwerpunkt auf die semantische Erfassung von natürlicher Sprache legen. Ebenfalls relevant sind Generierungsaufgaben, die sich auf die automatisierte Erzeugung von sprachlichen Inhalten beziehen. Des Weiteren lassen sich wissensintensive Aufgaben als bedeutsames Anwendungsfeld konstatieren, bei dem die Modelle dazu dienen, umfangreiche Wissensbestände zu verarbeiten und zu extrahieren. % Hier könnte ein Zitat zu den Anwendungsfeldern von LLMs eingefügt werden.

Die Vielfalt vortrainierter Sprachmodelle ermöglicht ihren Einsatz in unterschiedlichsten Kontexten\footcite{zhou2023comprehensive}. Dabei nimmt die Anzahl dieser Modelle kontinuierlich zu, was auf ein wachsendes Interesse und anhaltende Forschung in diesem Bereich hinweist \footcite{naveed2023comprehensive}. Ein konkretes Beispiel ist BERT, das erfolgreich für die Optimierung von Suchanfragen eingesetzt wird\footcite{devlin2018bert}. Durch die Anwendung solcher Modelle können komplexe linguistische Aufgaben effizient bewältigt werden, und sie spielen eine entscheidende Rolle in der Weiterentwicklung von NLP-Anwendungen. % Hier könnte ein Zitat zur Rolle von LLMs in der Weiterentwicklung von NLP-Anwendungen eingefügt werden.
