\section{Einleitung} \label{sec:einleitung}
Seit der Einführung des Chatbots ChatGPT von OpenAI im November 2022\footcite[Vgl.][]{ChatGPT-announcement} hat das Interesse der Allgemeinheit an künstlicher Intelligenz stetig zugenommen. \footcite[Vgl.][]{google_trends_ki}
So ist es nicht verwunderlich, dass auch andere Technologieanbieter wie Meta oder Google auf die große Nachfrage reagieren und ihre eigenen Sprachmodelle PaLM \footcite[Vgl.][S. 1]{anil2023palm} bzw. LLaMA \footcite[Vgl.][S. 1]{touvron2023llama} entwickeln.
Im Zuge dieser Entwicklung hat sich ein umfassender Forschungszweig herausgebildet, der sich mit der Frage beschäftigt, wie genau \acp{llm} miteinander verglichen werden können.
Diese Frage wird umso relevanter, da die Anwendungsgebiete von Sprachmodellen vielfältig sind und von Textgenerierung über maschinelles Übersetzen bis hin zum Zusammenfassen von Dokumenten reichen.
Die zentrale Herausforderung besteht darin, objektive Vergleichsparameter und Evaluierungsmethoden zu entwickeln, die es ermöglichen, die Leistungsfähigkeit dieser Modelle in verschiedenen Anwendungsszenarien zuverlässig zu beurteilen.
Deshalb soll diese Arbeit die Forschungsfrage aufgreifen, wie \acp{llm} effektiv miteinander verglichen werden können.

Diese Arbeit verfolgt daher zwei zentrale Ziele. Zum einen soll herausgearbeitet werden, inwiefern sich verschiedene \acp{llm} voneinander unterscheiden, um so eine Grundlage für die Beantwortung der Forschungsfrage zu schaffen.
Zum anderen soll Benchmarking allgemein als Methode zur Beurteilung der Leistungsfähigkeit von \acp{llm} vorgestellt werden.
Dabei liegt ein besonderer Fokus darauf, verschiedene spezifische Benchmarks einzuführen, um die Performance der Modelle zu messen und sie untereinander vergleichbar zu machen.

Dazu soll die Arbeit wie folgt gegliedert werden.
In Kapitel \ref{sec:sprachmodelle} soll zunächst ein Überblick über die grobe Funktionsweise (\ref{sec:sprachmodelle_allgemein}) und Anwendungsgebiete (\ref{sec:einsatzgebiete}) von \acp{llm} gegeben werden. Zusätzlich in Abschnitt \ref{sec:abgrenzung} gezeigt werden, inwiefern sich verschiedene \acp{llm} voneinander unterscheiden.
Mit diesem Wissen kann sich dann in Kapitel \ref{sec:benchmarking_konzepte} mit der eigentlichen Forschungsfrage beschäftigt werden.
Dabei soll zunächst Benchmarking als generelle Methode zur Beurteilung der Leistungsfähigkeit vorgestellt werden.
Anschließend sollen verschiedene heute relevante Benchmarks vorgestellt werden, die sich für die Beurteilung von \acp{llm} eignen.
Dabei sollen sowohl die Funktionsweise als auch Vor- und Nachteile der jeweiligen Benchmarks aufgezeigt werden.
Auch soll es hier darum gehen, vor welchen Herausforderungen Benchmarks generell stehen.
Kapitel \ref{sec:aktuelle_rankings} soll danach die in den vorherigen Kapiteln erarbeiteten Erkenntnisse zusammenfassen, indem mithilfe aktueller Rankings von \acp{llm} die Performance der Modelle anhand der vorgestellten Benchmarks verglichen wird.
Abschließend soll in Kapitel \ref{sec:schlussbetrachtung} ein Fazit gezogen werden, indem die Forschungsfrage beantwortet und ein Ausblick auf zukünftige Entwicklungen gegeben wird.

Die Arbeit an sich stellt also eine reine Literaturarbeit dar, die sich auf die Auswertung von bereits veröffentlichten Forschungsergebnissen stützt.
Dafür wurde Literatur genutzt, die sich mit der Funktionsweise von \acp{llm} beschäftigt, sowie mit der Frage, wie diese Modelle miteinander verglichen werden können, und bis Oktober 2023 veröffentlicht wurde.
Konkret wurden dafür die Datenbanken Google Scholar und die ACM Digital Library genutzt.
Es wurde der folgende Suchstring bei der ACM Digital Library verwendet:
\begin{verbatim}
    "query": {AllField:("large language model?") 
              AND AllField: ("benchmark?" "use case?" "performance")} 
    "filter": {E-Publication Date: (* TO 10/31/2023)},
              {ACM Content: DL}
\end{verbatim}
Dieser wurde entsprechend für Google Scholar angepasst.
