\section{Benchmarking-Konzepte}
Angesichts der breiten Palette von verfügbaren Modellen und den unterschiedlichen Anwendungsgebieten, die in Kapitel 2 eingehend erläutert wurden, erfordert die Entscheidung für ein bestimmtes Modell seitens Einzelpersonen oder Organisationen einen strukturierten Ansatz.
Dieser Ansatz sollte es ermöglichen, anhand einer gründlichen Bewertung spezifischer Anforderungen, Leistungsmetriken und des konkreten Anwendungskontexts eine informierte Entscheidung bei der Auswahl eines geeigneten \acp{llm} zu treffen.
Darum greift die wissenschaftliche Forschung auf das Benchmarking zurück, um verschiedene Modelle anhand unterschiedlicher Kriterien zu testen.
Benchmarking dient als methodischer Ansatz, um die Leistungsfähigkeit von \acp{llm} objektiv zu bewerten und miteinander zu vergleichen.
Hierbei werden spezifische Evaluierungskriterien und Testverfahren angewendet, um zu ermitteln, wie gut ein Modell in einem bestimmten Anwendungskontext abschneidet.
Im Folgenden sollen nun verschiedene Benchmarks für \acp{llm} vorgestellt werden.
Dabei ist es sinnvoll mit den älteren Benchmarks zu beginnen, da der Aufbau der Benchmarks in den letzten Jahren immer komplexer geworden ist.

