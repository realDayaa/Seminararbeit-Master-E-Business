\section{Schlussbetrachtung} \label{sec:schlussbetrachtung}
In dieser Arbeit wurde die Welt der Sprachmodelle und das Benchmarking dieser Modelle eingehend betrachtet.
Es wurde deutlich, dass die Bewertung und der Vergleich von Sprachmodellen mit Hilfe von Benchmarks entscheidend für das Verständnis ihrer Leistungsfähigkeit in verschiedenen Anwendungsszenarien sind.
Die Einführung verschiedener Benchmarks wie GLUE, SuperGLUE und SQuAD sowie die Diskussion ihrer Stärken und Schwächen hat gezeigt, wie komplex und vielfältig die Bewertung von Sprachmodellen sein kann.

Die verwendete Literatur, die für diese Arbeit herangezogen wurde, spiegelt den aktuellen Stand der Forschung zum Zeitpunkt der Erstellung dieser Arbeit wider.
Es ist jedoch wichtig zu betonen, dass das Gebiet der künstlichen Intelligenz und insbesondere der Sprachmodelle sehr schnelllebig ist.
Dies bedeutet, dass einige der verwendeten Quellen bereits zum Zeitpunkt der Veröffentlichung dieser Arbeit möglicherweise nicht mehr den neuesten Entwicklungen entsprechen.
Die rasante Entwicklung in diesem Bereich führt dazu, dass kontinuierlich neue Forschungsergebnisse veröffentlicht werden, die bestehende Annahmen und Ergebnisse herausfordern oder ergänzen können.

Es wurde auch hervorgehoben, dass trotz der fortschrittlichen Methoden und Techniken im Bereich der Sprachmodellierung Herausforderungen wie Überanpassung, mangelnde Repräsentativität und die dynamische Natur der Sprache bestehen bleiben.
Diese Herausforderungen unterstreichen die Notwendigkeit einer kontinuierlichen Entwicklung und Anpassung der Bewertungsmethoden.

Die Betrachtung der aktuellen Rankings von Sprachmodellen auf Plattformen wie Hugging Face gab interessante Einblicke in den aktuellen Stand der Technik. Es wurde jedoch auch klar, dass solche Rankings zwar hilfreich sind, aber mit Vorsicht interpretiert werden sollten, da sie nicht immer die Gesamtleistung eines Modells in allen realen Anwendungsszenarien widerspiegeln.

Zusammenfassend lässt sich sagen, dass das Feld der Sprachmodelle und ihr Benchmarking ein dynamischer und sich ständig weiterentwickelnder Bereich ist.
Die in dieser Arbeit diskutierten Konzepte und Methoden bieten eine solide Grundlage für das Verständnis der aktuellen Landschaft und legen nahe, dass die zukünftige Forschung und Entwicklung auf diesem Gebiet weiterhin von großer Bedeutung sein wird.
Es bleibt spannend zu beobachten, wie sich Sprachmodelle und ihre Bewertungsmethoden weiterentwickeln werden, um den sich ständig ändernden Anforderungen und Herausforderungen gerecht zu werden.
Die verwendeten Quellen stellen daher nur einen Ausgangspunkt dar, und es wird empfohlen, sich stets mit der neuesten Literatur und Forschung auseinanderzusetzen, um auf dem neuesten Stand zu bleiben.